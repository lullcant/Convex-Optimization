\vspace*{0.25cm}

\noindent\hrulefill

\thispagestyle{empty}

\begin{center}
\begin{large}
\sc{Stanford University \vspace{0.3em} \\Convex Optimization \\ By Stephen Boyd}
\end{large}

\hrulefill

\vspace*{5cm}
\begin{Large}
\sc{{Reading Notes}}
\end{Large}

\vspace{2em}

\begin{large}
\sc{{Convex Set
\vspace{0.5em}

}}
\end{large}
\end{center}
\vfill

\begin{table}[h!]
\flushleft
\begin{tabular}{lll}
Note writer: \textbf{Linda Wei} 

Date: 6 Jan. 2023

\end{tabular}
\end{table}

\hfill
\newpage
\tableofcontents
\setcounter{page}{0}
\thispagestyle{empty}
\newpage
\section{Affine Set and Convex Set}
In this section, a special definition of line and segment will be talked about.
\subsection{Line and Segment}
\begin{Definition}{1}{1}
Let $x_1 \neq x_2$ are two different point in a set $\textbf{R}^n$, if for any $\theta \in \textbf{R}$, we have the point of the format $y=\theta x_1 + (1-\theta )x_2$, then those $y$ compose a \textbf{line} through $x_1$ and $x_2$. If we constrain that the $\theta$ should range from 0 to 1, we call this a \textbf{segment}. The line and the segment are shown in the following figure.
From the figure (the deep dark means segment) follows we can view $y$ as the sum of the \textbf{base point} $x_2$ and the product of parameter $\theta$ and direction $(x_1-x_2)$.
\begin{figure}[h]
    \centering
    \includegraphics{Linda Wei work sheets/figure/line.JPG}
    \caption{Figure of a Line and Segment}
    \label{fig:my_label}
\end{figure}
\end{Definition}
\subsection{Affine Set}
\begin{Definition}{1}{2}
Given a set $C \subset R^n$, if $\forall$ two points $x_1,x_2 \in C$, the line go through  $x_1,x_2$ are still in C, we call $C$ is an \textbf{Affine Set}.
In mathematical languae. Given a set $C$, $\forall$ $x_1,x_2 \in C$, $\theta \in R$. we have $\theta x_1+ (1-\theta)x_2 \in C$.\\
\textbf{Remark}: $C$ is a set contains all the linear combination of $x_1,x_2$.
\end{Definition}

\begin{Definition}{1}{3}
The linear combination we mentioned above can be extended to multiple points. Given n points $x_1,x_2,\cdots x_n \in C$, and $\theta_1,\theta_2,\cdots,\theta_n \in R$. if $\sum_{i=1}^n \theta_i=1$, we call $\sum_{i=1}^n \theta_i x_i$ as the \textbf{affine combination} of $x_1,x_2,\cdots x_n$.\\
Remark: Using mathematical Induction, we can conclude that: an affine set will include all of the affine combination of the points in it, i.e. $\sum_{i=1}^n\theta_i x_i \in C$. 
\end{Definition}

\begin{Definition}{1}{4}
If C is an affine set, and $x_0 \in C$, then the set $V=\{v\mid x-x_0, x\in C\}$ will be a subspace. We call it the \textbf{subspace} of affine set $C$. And we define the \textbf{dimension of the affine set} as the dimension of the subspace $V$,i.e. $dim(C)=dim(V)$. 
\end{Definition}

\begin{Proof}{1}{4}
For $a,b\in R,a+b=1$ and $v_1,v_2 \in V$. We have:\\
$$
\begin{aligned}
av_1 + bv_2 &= a(x_1-x_0)+b(x_2-x_0)\\
&=ax_1+bx_2-x_0\\
&\because ax_1+bx_2 \in C\\
&\therefore ax_1+bx_2-x_0 \in V.\\
\text{thus V is a subspace.}
\end{aligned}
$$
\end{Proof}
